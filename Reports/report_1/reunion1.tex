\documentclass[a4paper,12pt]{article}

\usepackage[utf8]{inputenc}
\usepackage[T1]{fontenc}
\usepackage{lmodern}
\usepackage{amsmath} % For math environments
\usepackage{amssymb} % For math symbols
\usepackage{hyperref} % For hyperlinks
\usepackage{tikz}

\title{Reunion report 1}
\author{Simon Sepiol-Duchemin Joshua Setia}
\date{\today}

\begin{document}

\maketitle

\section{Recap of the reunion}

\subsection*{Studying roots of a polynomial on different intervals}

We described a method allowing us to study the positive real roots of a
polynomial on different intervals. Here are the steps :
\begin{itemize}
  \item Bound the maximum real positive root with \( Bound(\underline{f}) = B = 2^k\)
  \item Substitute \(x\), allowing us to change the interval of study\\ex : $\quad$ \(x \rightarrow \frac{x}{2^k} \quad \Rightarrow \quad ]0,B[ \rightarrow ]0,1[\)
  \item Factorize \(f\) in order to have integer coefficients again \\\(f \in \mathbb{Q}[x] \rightarrow \tilde{f} \in \mathbb{Z}[x]\)
  \item Use Descartes’ rule of sign to determine the number of real positive roots in the new interval
\end{itemize}


\subsection*{Recursive method}
Using the method described above to study the roots on different intervals, we want to do it on specific intervals repeatedly until we've successfully isolated each root :
\begin{itemize}
  \item \(]0,1[\) by doing the substitution \(x \rightarrow \frac{x}{2^k}\)
  \item \(]0,+\infty[\) by doing the substitution \(x \rightarrow \frac{1}{y+1} \quad (y=\frac{1}{x}-1\))
  \item \(]0,\frac{1}{2}[\) by doing the substitution \(x \rightarrow 2y \quad (y=\frac{x}{2}\))
  \item \(]\frac{1}{2},1[\) by doing the substitution \(x \rightarrow 2y-1 \quad (y=\frac{x}{2}+\frac{1}{2}\))
\end{itemize}

\subsection*{Role of the Taylor Shifts}
When doing the substitution \(x \rightarrow \frac{1}{y+1}\), we will only need to perform a Taylor Shift, since doing \(f(\frac{1}{x})\) does not require any operations. Indeed, for a polynomial \(f = (f_0,....,f_d)\) :
\[
  f(\frac{1}{x}) = \frac{f_0x^d + f_1x^{d-1} + .... + f_d}{x^d}
\]

\subsection*{Trivial cases}
For specific values of x, verifiyng the number of real positive roots does not require any operations :
\begin{itemize}
  \item \(x = 0\), we only need to know if the least significant coefficient is zero
  \item \(x= 1\), we only need to summ the coefficients
\end{itemize}

\section{Tasks for next reunion}

\subsection*{Implementations}
Test flint's polynomial multiplication performance.\\
The tested polynomials must have integers coefficients, be univariate and dense (nonzero coefficients).
Measure the performance by changing :
\begin{itemize}
    \item the degree, with fixed coefficients size
    \item the coefficients size (up to thousands of bit), with fixed degree
\end{itemize}
Deduce which algorithms are used for the flint polynomial multiplication operation (naive, Karatsuba, Cantor and Kaltofen...).

\subsection*{Search and suggest}
\begin{itemize}
  \item Function \(Bound(\underline{f})\) for bounding a polynomial's real positive roots
  \item Fast way of computing \(f(\frac{1}{2})\), similar to \(f(1)\) and \(f(0)\), only using shifts
\end{itemize}

\subsection*{Understand, learn and be able to redo on board}
\begin{itemize}
    \item Decartes' rule of sign's proof
    \item Divide and conquer algorithm for Taylor Shifts
\end{itemize}

\end{document}