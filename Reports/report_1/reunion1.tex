\documentclass[a4paper,12pt]{article}

\usepackage[utf8]{inputenc}
\usepackage[T1]{fontenc}
\usepackage{lmodern}
\usepackage{amsmath} % For math environments
\usepackage{amssymb} % For math symbols
\usepackage{hyperref} % For hyperlinks
\usepackage{tikz}

\title{Reunion report 1}
\author{Simon Sepiol-Duchemin Joshua Setia}
\date{\today}

\begin{document}

\maketitle

\section{Recap of reunion}

\subsection*{Studying roots of a polynomial on different intervals}

For a polynomial (written as an array) \( f = (f_0,....,f_d) \in \mathbb{Z}^{d+1} \), we can bound its maximum real positive root with a function \( Bound(\underline{f}) = B = 2^k\) (with B a power of 2).\\
By performing operations on \(x\), we can reduce the interval bounding the roots \([0,B]\) to a new interval.\\
\\
For example, by substituting \(x\) by \(\frac{x}{2^k}\), we'll then study the real positive roots on the interval \([0,1]\).
The corresponding polynomial will be \(f(\frac{x}{2^k}) \in \mathbb{Q}[x]\). We will then need to factorize \(f\) to have \(\tilde{f} \in \mathbb{Z}[x]\), in order to have integers coefficients.

\subsection*{Recursive method}
Using the method described above to study the roots on different intervals, we want to do it on specific intervals repeatedly until we've successfully isolated each root :
\begin{itemize}
  \item \(]0,1[\) by doing the substitution \(x \rightarrow \frac{x}{2^k}\)
  \item \(]0,+\infty[\) by doing the substitution \(x \rightarrow \frac{1}{y+1}\)
  \item \(]0,\frac{1}{2}[\) by doing the substitution \(x \rightarrow 2x\)
  \item \(]\frac{1}{2},1[\) by doing the substitution \(x \rightarrow 2x\)
\end{itemize}

\subsection*{Role of the Taylor Shifts}
When doing the substitution \(x \rightarrow \frac{1}{y+1}\), we will only need to perform a Taylor Shifts, since doing \(f(\frac{1}{x})\) does not require any operations. Indeed, for a polynomial \(f = (f_0,....,f_d)\) :
\[
  f(\frac{1}{x}) = \frac{f_0x^d + f_1x^{d-1} + .... + f_d}{x^d}
\]

\subsection*{Trivial cases}
For specific values of x, verifiyng the number of real positive roots does not require any operations :
\begin{itemize}
  \item \(x = 0\), we only need to know if the least significant coefficient is zero
  \item \(x= 1\)
\end{itemize}

\section{Tasks for next reunion}

\subsection*{Implementations}
Test flint's polynomial multiplication performance.\\
The tested polynomials must have integers coefficients, be univariate and dense (nonzero coefficients).
Measure the performance by changing :
\begin{itemize}
    \item the degree, with fixed coefficients size
    \item the coefficients size (up to thousands of bit), with fixed degree
\end{itemize}

\subsection*{Search and suggest}
\begin{itemize}
  \item Function \(Bound(\underline{f})\) for bounding a polynomial's real positive roots
  \item Fast way of computing \(f(\frac{1}{2})\), similar way to \(f(1)\) and \(f(0)\), only using shifts
\end{itemize}

\subsection*{Understand, learn and be able to redo on board}
\begin{itemize}
    \item Decartes' rule of sign's proof
    \item Divide and conquer algorithm for Taylor Shifts
\end{itemize}

\end{document}